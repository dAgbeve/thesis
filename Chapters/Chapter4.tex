\chapter{Conclusion}
Simulating a novel medium access control (MAC) protocol, APT-MAC - an adaptive
protocol to the data needs of the nodes accessing the medium, on the NS-3 platform
was the goal of this experiment. The aim was extended to investigate how the
performance of APT-MAC compare to that of TDMA.\\\\
The algorithm of the data-requirement-adaptive protocol is based on a solution to
the Multi-arm Bandit problem, more precisely $\varepsilon$-greedy algorithm, hence
the opening of this documentation sought to shed light on the Multi-arm Bandit
problem with its $\varepsilon$-greedy algorithm solution. A brief explanation of
TDMA was also introduced.
Next, was a description of the actual simulation, the various classes and the
task of the various functions used in the simulation. Three categories of 
tag-augmented data-packet producing sensory nodes were replicated, namely -
periodic, event-based and real-time. Both readers, that implementing TDMA and
APT-MAC respectively. The focus was turn on measuring the performance of both
TDMA and APT-MAC and an elucidation on how the variables of interest - data loss
and packet delay - were harnessed.\\\\
The results clearly showed that the performance outcome, that is, the percentage of
new data that is lost by the sensory devices and the speed, in terms of time, with
which a newly generated packet is delivered to reader due to the readers inability
or ability to query the tags on time is much better with APT-MAC.
In the worse case scenario, data loss and packet delay of APT-MAC were less than
$1\%$ and $0.9$\textit{s} respectively, and that of TDMA stood at $7.05\%$ for
data loss and almost $2$\textit{s} with TDMA.
\section{Future Work}
A follow up on this investigation could focus on the effects of mobility of the
tag-augmented sensory devices on the performance of the protocol. There are other
solutions to the Multi-Arm bandit problem, chief among them is the Upper Confidence
Bounds \cite{Sutton&Barto}. An investigation into which other algorithm could
perform better than $\varepsilon$-greedy algorithm is the focus of a future work
on this.


