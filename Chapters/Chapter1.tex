\chapter{Introduction}

The later part of the past decade has seen a tremendous increase in the research of Internet of Things (IoT) devices mostly geared towards energy efficiency.
This is in part, due to the fact that IoT devices have an underlining positive factor of having to save energy usage and hence must be design as not excessively use energy in itself.
Not only do these interconnected devices provide comfort but have also become critical parts of our daily lives: saving lives, expediting interactions and transactions.
Another factor that fueled the advancement in the research of energy efficient IoT devices is the enormous progress made in the field of Machine learning specifically Reinforcement learning.\\\\
Maximizing a numerical reward signal by learning which actions to take in a given situation is what Reinforcement learning is about.
The actions to take by the learning agent pertaining to various situations are not preprogrammed into the learner, but instead this is discovered by taking the actions that maximize the reward margin\cite{Sutton&Barto}. This works in the cycle of sense-action-goals and learning is from immediate interactions with the environment.\\\\
The APT-MAC protocol, which is the focus of this paper, seeks to solve the problem of the reliance on battery.
The protocol utilizes the Multi-Arm Bandit algorithm, a reinforcement learning algorithm, in tandem with Radio-frequency identification (RFID) signals to enable battery-free communication of wireless devices\cite{Maselli}.\\\\
The aim of this work is to simulate the APT-MAC protocol in NS3 and in comparison to the static TDMA protocol.

\section{The Protocols}
This section deals with the description of the protocol: APT-MAC and Time Division Multiple Access (TDMA).

\subsection{TDMA}




\subsection{A Subsection}


\section{Another Section}


